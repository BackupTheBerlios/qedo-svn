\documentclass[12pt,a4paper]{report}

\begin{document}


\title{Qedo - Installation}
\author{Harald Boehme \\ Bertram Neubauer \\ Tom Ritter \\ Frank Stoinski}

\maketitle

\setcounter{page}{1}

\tableofcontents


\chapter{Introduction}
\label{sec:Introduction}

The Qedo Run-Time and the Qedo Code Generator are based upon other software packages.
This installation guide refers to {0.7.2} version of Qedo software.

...

\section{Contact}
\label{sec:Contact}

If you have any problem or if you have any comments do not hesitate to contact the other Qedo users by using the qedo-devel mailing list. You can reach it at qedo-devel@lists.berlios.de. You can also use the bug tracking system provided by the Qedo project. You can reach it at \verb http://www.qedo.org . this site has all relevant information about the Qedo project. In any case you can also contact the authors directly. The email addresses are listed on the Qedo web page as well.


\chapter{Installation of third party packages}
\label{sec:Installation}

This installation guide will describe the installation process for Unix and Windows based systems.

This following sections describes the process of installing the Qedo. This includes the Code Gnerator and the run-time. Because Qedo is a software that solve complex problems Qedo itself needs many other software products. You have to install all of them before you can compile Qedo. This will take you some time and experiences with installing such software would be very helpful. Nevertheless, we hope that this little description will give enough information to accomplish all needed steps.
The Qedo software can be compiled by using the MS Visual Studio 6 (deprecated, no longer supported), MS Visual Studio.Net 2002 (VC7.0), Solaris gcc, or Linux gcc.

\section{ORB}
\label{sec:ORB}
The Qedo project is designed to work with arbitrary ORB implementations which support at least a basic set of features. This set comprises OBV, Portable Interceptors, ...

Anyway, for each ORB implementation some minimal changes have to be done. Currently ORBacus 4.1 and MICO are supported by deafault. The TAO support is deprecated. In order to install Qedo a supported ORB implementation has to be installed on the target host. Pay attention that it has to be dynamically linked and the linker can find the according libraries. At the moment Qedo generator supports 3 ORBs MICO, TAO and ORBacus. For using one of the ORBs you need to install the ORB and to set an appropriate environment variable. Follow the instructions coming with each ORB to install it properly. We will only point out some important details.

\subsection{ORBacus}
\label{sec:ORBacus}
To install ORBACUS you have to do at least the following steps.

\begin{itemize}
    \item Obtain the ORBacus Software (version 4.1.1) from \\
    \small
    \verb http://www.iona.com/products/orbacus_home.htm .
    \normalsize

    \item Configure ORBacus by editing the file \\ \verb /config/Makefile.rules.mak  set the prefix and use \verb DEBUG=yes  and \verb DLL=yes .

    \item compile ORBacus by calling
    \begin{verbatim} nmake -f Makefile.mak install_min}
    \end{verbatim}

    \item set the environment by defining the following vars.
    \begin{verbatim}
    ORBACUS = (e.g. C:\OOC)
    ORBACUS_CONFIG= %ORBACUS%\ooc\ini\orbacus.ini
    \end{verbatim}

    \item Create an ini directory somewhere in your filesystem, preferably in your home directory or in the ORBACUS installation. Create an ini file in this directory. The path to this ini file must correspond to the \verb ORBACUS_CONFIG  var. This ini file must contain lines similar to this:
    \small
    \begin{verbatim}
            ooc.orb.service.NameService = corbaloc:iiop:127.0.0.1:3000/NameService
            ooc.naming.endpoint=iiop --port 3000
    \end{verbatim}
    \normalsize

    \item add the following directories to your path and to the setting of you visual studio (executable files)
    \begin{verbatim}
    %ORBACUS%\lib
    %ORBACUS%\bin
    \end{verbatim}

\end{itemize}


\subsection{MICO}
\label{sec:MICO}
The currently supported version of MICO is 2.3.11.
%You need to install the latest snapshot in order to get Qedo configured correctly.

\noindent To install MICO you have to do at least the following steps.
\begin{itemize}
    \item Obtain the software from \verb http://www.mico.org .

    \item Install the software by following the installation steps provided by MICO. You need to install a multi-threaded version. The configuration of MICO on UNIX System differs from Windows based systems. On a UNIX based systems you need to call the configure script with appropriate options. A typical list of option should be as follows.
    \small
     \begin{verbatim}
     ./configure --prefix=/opt/mico --enable-threads
         --enable-pthreads
     \end{verbatim}
    \normalsize
    Calling  \verb make \verb install  will compile plain version of MICO. An option might be to use the \verb --disable-coss  this will make the compilation of MICO a lot more faster but you need to compile at least the Name Service manually because Qedo depends on it.

On Windows based systems you need to configure the \verb MakefVars.win32  file. Use the \verb #define  \verb RELEASE_BUILD  \verb =  \verb 1  option to decide whether to build an release or Debug version of MICO. Later on you need to activate the Threading support by uncommenting the \verb HAVE_THREADS  define. You need to configure the other threading related defines. Follow the instructions of the \verb MakeVars.win32  file to install the pthread library and to configure MICO correctly. We strongly recommend to use the second option of the pthread library \verb __CLEANUP_C .  After saving the file you need to call
    \begin{verbatim}
    nmake /f Makefile.win32
    \end{verbatim}

    to compile the MICO ORB.

    \item On a Unix system you need to be sure the MICO \verb bin  directory is included in the path before you can configure Qedo. The Qedo configure script will use a script of the MICO installation to obtain the options MICO is compiled with. On Unix Systems you need to add the \verb win32-bin  directory to the PATH environment setting to use MICO. For compiling Qedo with MICO in the Visual Studio you need to add this path the settings of the visual studio \verb Tools  \verb ->  \verb Options  \verb ->  \verb Projects  \verb ->  \verb VC++Dirs .

    \item We do not ecourage you to use a \verb .micorc  file, since some of the configuration you can make contradicts to the ones of the Qedo.conf file. This is true in particular for the initial NameService reference. If you are sure what you are doing, you can use the \verb .micorc file for other MICO configuration issues. After compiling and installing MICO, you can create a .micorc file. This file should be in the home directory or in the current working directory or you can use an environment setting pointing to this file (MICORC). Consult the MICO documentation for details.

\end{itemize}

\subsection{TAO (deprecated)}
\label{sec:TAODeprecated}

TAO comes with the ACE Framework (Version 5.3).
To install TAO you have to do at least the following steps.
\small
\begin{verbatim}
obtain software:
    Version TAO 1.3 ACE 5.3
    get from: http://www.cs.wustl.edu/~schmidt/TAO.html
install/compile:
    ACE_ROOT = C:\ACE_WRAPPERS
    TAO_ROOT = C:\ACE_WRAPPERS\TAO
    TAO      = C:\ACE_WRAPPERS
    create a config.h in %ACE_WRAPPERS%\ace:
        #define ACE_HAS_STANDARD_CPP_LIBRARY 1
        #include "ace/config-win32.h"
    use the TAO workspace provided at %ACE_ROOT%\tao to compile TAO
    there are a lot of project you need at least to compile these ones:
        TAO_IDL
        PortableServer
        DynamicAny
        DynamicInterface
        NameService
        IFR_Client
    add the following to the path variable
        %ACE_ROOT%\bin
\end{verbatim}
\normalsize

The Qedo Runtime worspace supports configurations for 3 ORBs (ORBACUS, MICO, TAO). Each of these ORBs has specific characteristics (e.g the names of the skeleton files). These different characteristics make it really difficult to integrate the configurations into a single workspace. There are some serious problems with TAO. To overcome these problems you have to do the following things.
\small
\begin{verbatim}

copy %ACE_ROOT%\TAO\tao\IFR_Client\IFR_BasicC.h
    %Qedo%\ComponentIDL\IFR_Basic.h

copy %ACE_ROOT%\TAO\tao\IFR_Client\IFR_BasicC.h
    %Qedo%\ComponentIDL\IFR_Basic_skel.h

copy %ACE_ROOT%\TAO\tao\PortableServer\PortableServerC.h
    %Qedo%\ComponentIDL\PortableServer.h

copy %ACE_ROOT%\TAO\tao\PortableServer\PortableServerC.h
    %Qedo%\ComponentIDL\PortableServer_skel.h

copy %ACE_ROOT%\TAO\orbsvcs\orbsvcs\CosNamingC.h
    %Qedo%\ComponentIDL\CosNaming.h
copy %ACE_ROOT%\TAO\orbsvcs\orbsvcs\CosNamingC.h
    %Qedo%\ComponentIDL\CosNaming_skel.h

modify the file: %ACE_ROOT%\TAO\tao\Current.pidl

add at beginning of the file:
#ifndef _CURRENT_PIDL_
#define _CURRENT_PIDL_

add at the end of the file

#endif

\end{verbatim}
\normalsize

\section{Cygwin}
\label{sec:CYGWIN}
For several tasks in the process of compiling Qedo some tools are needed (e.g. zip, flex, bison). Those tools are usually not available on windows systems though you can use cygwin. Cygwin offers a variety of unix tools for Windows systems. You can obtain the latest version of Cygwin from \verb http://www.cygwin.com . For most cases a default installation plus complete devel package and zip utility should be sufficient.
Sometimes, problems with WinCVS and installed tcl are reported. If you use WinCVS you should avoid installing tcl.

\section{XERCES-C}
\label{sec:XERCESC}

The Qedo project furthermore requires the xerces-c XML library version 2.3 to handle the deployment XML descriptors. On Unix based system is sometimes a Xerces-c version installed. You need to be sure this xercex installation is of the right version. For installing Xerces you need to do the following steps.

\begin{itemize}
    \item Obtain Software from
    \verb http://xml.apache.org/dist/xerces-c/stable/ .
    \item Follow the installation instruction of this xerces package.
    \item On Windows based systems you need to set environment Var
    \begin{verbatim}
    XERCES=<YOUR_XERCES_INSTALLATION_DIR>
    \end{verbatim}
\end{itemize}

\section{ZLib}
\label{sec:ZLib}


According to the CCM specification the Qedo project makes use of software packaging. In order to install Qedo the zlib 1.1.4 or a later version has to be installed on the target host. Usually this is installed on most Unix based systems.


\begin{itemize}
    \item Obtain Software from \verb http://www.gzip.org/zlib/ . For Windows based systems you can use the pre-compiled versions. For Unix based systems you need to follow the installation instructions.
    \item On Windows based systems you need to set the environment var \verb ZLIB  pointing to the installation directory of the ZLib package.
    \item On Windows based systems you need to add the directory \verb $(ZLIB)/dll32  to your path environment setting.
\end{itemize}
Automatic packaging under Windows requires the zip tool from cygwin. Make sure, it is in your path.


\section{Kimwitu++}
\label{sec:Kimwitu}

Kimwitu++ is used in the front-end of the code generator. You only need to install it if you want to compile the code generator. The currently supported version is 2.3.4. You need to do the following in order to install kimwitu++.

\begin{itemize}
    \item Obtain Software from

    \verb http://site.informatik.hu-berlin.de/kimwitu++/download.html . On Windows based systems you can use the pre-compiled version of kimwitu++. On Unix based systems you should compile it on you own.
    \item On Unix based systems kimwitu++ is compiled by calling \verb configure  and \verb make  . Consult the Installation guide of kimwitu++ for further details. The result of the compilation process is a single binary.
    \item Add the kimwitu++ Binary to the path setting.
\end{itemize}

\section{wxWidgets}
\label{sec:WxWidgets}

wxWidget is an open source C++ GUI framework to make
cross-platform programming. In former times wxWidget was devoloped
under the name wxWindows. After a polite request from Microsoft,
wxWindows was renamed in wxWidget. The currently supported version
of wxWidget is 2.4.2. You need to install it if you want to
compile the \verb qcontroller .

\begin{itemize}
    \item Obtain Software from \verb http://www.wxwidgets.org .

    \item Consult the Installation guide of wxWindows for further
    details
 
\end{itemize}

\chapter{Installation of Code Generator}
\label{sec:InstallationOfCodeGenerator}

\section{Compiling on Windows}
\label{sec:CompilingOnWindows}

First of all you need to define an environment setting to specify where the compiled code generator shall be installed. For that reason you need to define the \verb QEDO  environment setting. It is a good idea to re-start an already running Visual Studio to be sure this change is propagated.

If you use the MS Visual Studio 6 you have to open the workspace:
\begin{verbatim}
/build/win32/vc6/generator.dsw
\end{verbatim}

But the VC6 workspaces are not maintained any more.

If you use the MS Visual Studio .Net 2003 you have to open the solution file:
\begin{verbatim}
/build/win32/vc7/generator.sln  solution File.
\end{verbatim}

Then you will have to set the active configuration.

\begin{verbatim}
mico    ->   Win32 Debug_mico
tao     ->   Win32 Debug_tao
orbacus ->   Win32 Debug_orbacus
\end{verbatim}

Then you need to build the whole solution. This will automatically install the code generator (\verb cidl_gen ) in the bin directory under the Qedo installation directory pointed to by the \verb QEDO  environment var.

\section{Compiling on Unix}
\label{sec:CompilingOnUnix}

For the details of the configure script see next section.


\chapter{Installation of Run time}
\label{sec:InstallationOfRunTime}


It might be a good idea to remove older versions of Qedo before installing a new one. The installation into the same directory may cause problems.


\section{Compiling on Windows}
\label{sec:CompilingWindows}
First of all you need to define an environment setting to specify where the compiled Qedo shall be installed. For that reason you need to define the \verb QEDO  environment setting. Preferably you should use the same setting as used for the code generator. It is a good idea to re-start an already running Visual Studio to be sure this change is propagated.

If you use the MS Visual Studio 6 you have to open the workspace \verb Qedo.dsw . But the VC6 workspaces are not maintained currently.

If you use the MS Visual Studio .Net 2002 you have to open the \verb StreamContainer.sln  solution File.

Then you need to choose the right configuration.

\begin{verbatim}
mico    ->   Win32 Debug_mico
tao     ->   Win32 Debug_tao
orbacus ->   Win32 Debug_orbacus
\end{verbatim}

Then you need to build the whole solution. This will automatically install the compiled executables in the directory pointed to by the \verb QEDO  environment var. Furthermore all lib and include file will also be installed under the \verb QEDO  directory.

\section{Compiling on Unix}
\label{sec:compilingUnix}

For Unix based systems Qedo support the configure, make install pattern. In the Qedo directory you can call \verb configure . This script supports currently the following optional features. 

\begin{verbatim}
--prefix=QEDOINSTALLDIR
\end{verbatim}

The option determins the installation directory.

\begin{verbatim}
--enable-debug
\end{verbatim}

The debug option is set by default.

\begin{verbatim}
--enable-streams
\end{verbatim}

The Streams feature is set by default.

\begin{verbatim}
--diable-qos
\end{verbatim}

The compilation of QoS feature is disabled by default

\begin{verbatim}
--enable-db
\end{verbatim}

The database integration is disabled by default.


The following optional packags are supported by the configure script.

\begin{verbatim}
--with_uuid=dir
\end{verbatim}

This option provides the path to the uuid library if it is not installed in the standard paths.

\begin{verbatim}
--with-xerces-c=dir
\end{verbatim}

This option provides the path to the xerces-c installtion if it is not installed in the standard paths.

\begin{verbatim}
--with-mico=dir
\end{verbatim}

This option provides the path to the mico installation if it is not installed in the standard paths.

\begin{verbatim}
--with-orbacus=dir
\end{verbatim}

This option provides the path to the orbacus installation if it is not installed in the standard paths.

\begin{verbatim}
--with-wxGTK=dir
\end{verbatim}

This option provides the path to the WxWidgets installation if it is not installed in the standard paths.

If the configure script complains about missing or wrong xerces. You can also try to check whether you xerces installation needs special libs. If this is the case you can use the following command to work around this problem. The following will solve a problem with a missing pthread lib.

\begin{verbatim}
LDFLAGS=-lpthread ./configure
\end{verbatim}

Please review the output of the configure script carefully since it may point you to some mistakes in configuration, which lead not to an obvoice compilation error.

After configuring Qedo you only need to call:
\begin{verbatim}
make install
\end{verbatim}

to build and install the Qedo run-time.

\chapter{Creating a Distribution}
\label{sec:CreatingADistribution}
This section explains in a very short way how to build a distribution package for Qedo. 

\begin{itemize}
    \item Install and possibly compile all needed software. Then you need to compile \verb genertor2  and \verb qedo  package. Make sure you compile the right configuration. All variables need to be set before. In particular you need to set the \verb Qedo  variable. This points to the place where all compiled Qedo sources a copied to. This directory will be used later to create the distribution package.
    
    \item Copy \verb micocossxxx.dll  , \verb micoxx.dll  and \verb idlxxx.dll  to \verb Qedo/bin .
    
    \item Copy the contents of the include directory of \verb MICO  to \verb Qedo/include/mico  directory. You can leave out \verb ministl  and \verb pocketpc .
    
    \item Edit \verb Components.h  \verb QedoComponents.h  and  \verb CORBADepends.h  . There are a number of full-path include statements which would point to the wrong directory. (Depends on the installation directory of the target machine.)
    
    \item Copy header files \verb pthreads.h  \verb sched.h  and \verb semaphore.h  from \verb pthreads  to \verb Qedo/include/pthreads .
    
    \item Copy \verb pthreadVC.lib  to \verb Qedo/lib  directory.
    
    \item Copy \verb pthreadsVC.dll  to \verb Qedo/bin  directory.
    
    \item test

\end{itemize}


\chapter{InstallShield}
\label{sec:InstallShield}
This chapter will describe the installation process using the InstallShield. This is can only be used on Windows based systems.
\end{document}
