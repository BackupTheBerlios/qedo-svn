\documentclass[12pt,a4paper]{report}


\begin{document}


\title{Qedo Run-Time Environment - Documentation}
\author{Harald Boehme \\ Bertram Neubauer \\ Tom Ritter \\ Frank Stoinski}

\maketitle

\setcounter{page}{1} 

\tableofcontents


\chapter{Introduction}
\label{sec:Introduction}

The Qedo Run-Time.

...
\section{Installation}
\label{sec:Installation}
For the Installation of the run-time consult the Installation Guide.


\section{Contact}
\label{sec:Contact}

If you have any problem or if you have any comments do not hesitate to contact the other Qedo users by using the qedo-devel mailing list. You can reach it at qedo-devel@lists.berlios.de. You can also use the bug tracking system provided by the Qedo project. You can reach it at http://qedo.berlios.de. this site has all relevant information about the Qedo project. In any case you can also contact the authors directly. The email addresses are listed on the Qedo web page as well.




\chapter{Architecture}
\label{sec:Architecture}
Qedo run-tim has the following architecture.

 ...

\section{Component Server}
\label{sec:AComponentServer}
The component server. 

 ... 

\section{Component Container}
\label{sec:ComponentContainer}
The Component Container. 

...

\section{Component Installer}
\label{sec:ComponentInstaller}
The Component Installer.

...

\chapter{Usage}
\label{sec:Usage}
This section explain the options. ...

\section{Start up}
\label{sec:StartUp}
To start the tun-time environment of Qedo you need to start a number of processes. 

\subsection{Name Service}
\label{sec:NameService}

The first of all is a name service. This name service needs to be started once for a Qedo deployment domain. Such a domain consist of a number of computers where distributed systems can be installed. How to start a name service depends on the used Name Service implementation. But you need to be sure to know the host where the name service is running and on what port the name service is listening. This information is needed for the other servers. Currently we use orb depend mechanism to promote the name service information to the other Qedo servers. E.g. if you are using MICO you need to create a .micorc file. this file need to contain the following line:
\small
\begin{verbatim}

  -ORBInitRef NameService=corbaloc::<hostname>:<port>/NameService
  
\end{verbatim}
\normalsize

To start a name service on a specific port usually works with the following option:

\small
\begin{verbatim}

  nsd -ORBIIOPAddr inet:<hostname>:<port>

\end{verbatim}
\normalsize

\subsection{Home Finder}
\label{sec:HomeFinder}

The HomeFinder is an optional Feature which can be used to find running homes. If a HomeFinder is available it is used. If a HomeFinder is not available a warninf message would be displayed but the creation of homes continues. The Homefinder is started by calling

\small
\begin{verbatim}

  homefinder

\end{verbatim}
\normalsize

\subsection{Component Installation}
\label{sec:ComponentInstallation}
The ComponentInstallation server is an implementation of the ComponentInstallation interface defined by CCM. This server is used to install the binaries of the components. This server is mandatory. To start this server you need to call

\small
\begin{verbatim}

  ci

\end{verbatim}
\normalsize

\subsection{Component Server Activator}
\label{sec:ComponentServerActivator}

The ComponentServerActivator server is an implementation of the correspoding interface defined by CCM. This server is used to create new component server and to manage them. This server is mandatory. To start this server you need to call 

\small
\begin{verbatim}

  csa

\end{verbatim}
\normalsize

\subsection{Assembly Factory}
\label{sec:AssemblyFactory}

This server implements the assembly factory and assembly interface defined be CCM. This server is mandatory if you like to use the automatic deployment features of CCM. You can also create component servers and containers by yourself. In this case you do not need this server. To start the server you need to call:

\small
\begin{verbatim}

  assf

\end{verbatim}
\normalsize

\subsection{Component Server}
\label{sec:ComponentServer}

The ComponentServer is started by the ComponentServerActivator automatically.


\chapter{Deployment}
\label{sec:Deployment}
The subdirectory Deployment of the Qedo installation directory contains all deployment related stuff. The XML file DeployedComponents.xml contains information about all installed component implementations and is used to keep those information persistent. The subdirectory ComponentPackages contains the softpackages for component implementations and assemblies. Currently for each assembly or component intended to be installed, the package has to be put there before manually. During component installation each component softpackage is unpacked into a temporary directory named by the UUID of the installed component implementation. These directories are created in the Runtime/ComponentImplementations subdirectory. Subsequently all generated servant code is also put into the Runtime/ComponentImplementations directory. The makefiles in this directory are used to generate servant code for the components. For each ORB implementation and architecture an own makefile
has to be provided. In future it will be determined by the software package descriptor elements 'os' and 'dependency'.

In order to deploy an assembly, a zip file has to be created by the assembly developer, containing an assembly descriptor, zip files for each component  and probably property file descriptors for component instances. The component zip files in turn contain a software package descriptor, the dynamic library for the components business logic and the idl file for the servant code generation.

\chapter{API Reference}
\label{sec:APIReference}

This section explains important part of the API

\section{Context}
\label{sec:Context}

The Session context


\section{container interface}
\label{sec:containerInterface}
The container interface

\end{document}

