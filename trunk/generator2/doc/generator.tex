\documentclass[12pt,a4paper]{report}


\begin{document}


\title{Qedo Code Generation Tool - Documentation}
\author{Harald Boehme \\ Bertram Neubauer \\ Tom Ritter \\ Frank Stoinski}

\maketitle

\setcounter{page}{1} 

\tableofcontents


\chapter{Introduction}
\label{sec:Introduction}

The Qedo Code generation Tool (code generator) is used to read model information and to produce files that are needed to create CORBA components on top of Qedo.

...

\chapter{Architecture}
\label{sec:Architecture}
The Code generator is divided into 3 parts. It is the front-end, the middle-end, and the back-end.

 ...

\section{Front-end}
\label{sec:FrontEnd}
The front end reads model information.

 ... 

\section{Middle-end}
\label{sec:MiddleEnd}
The middle-end is the model repository. 

...

\section{back-end}
\label{sec:backEnd}
The back-end produces the code.

...

\chapter{Usage}
\label{sec:Usage}
This section explain the options. ...
\end{document}

