\documentclass[12pt,a4paper]{report}


\begin{document}


\title{Qedo Code Generation Tool - Documentation}
\author{Harald Boehme \\ Bertram Neubauer \\ Tom Ritter \\ Frank Stoinski}

\maketitle

\setcounter{page}{1} 

\tableofcontents


\chapter{Introduction}
\label{sec:Introduction}

The Qedo Code generation tool (code generator) is used to read model information and to produce files that are needed to create CORBA components on top of Qedo.

...
\section{Installation}
\label{sec:Installation}
For the Installation of the run-time consult the Installation Guide.


\section{Contact}
\label{sec:Contact}

If you have any problem or if you have any comments do not hesitate to contact the other Qedo users by using the qedo-devel mailing list. You can reach it at qedo-devel@lists.berlios.de. You can also use the bug tracking system provided by the Qedo project. You can reach it at \verb http://www.qedo.org . this site has all relevant information about the Qedo project. In any case you can also contact the authors directly. The email addresses are listed on the Qedo web page as well.



	 
\chapter{Architecture}
\label{sec:Architecture}
The Code generator is divided into 3 parts. It is the front-end, the middle-end, and the back-end.

 ...

\section{Front-end}
\label{sec:FrontEnd}
The front end reads model information.

 ... 

\section{Middle-end}
\label{sec:MiddleEnd}
The middle-end is the model repository. 

...

\section{back-end}
\label{sec:backEnd}
The back-end produces the code.

...

\chapter{Usage}
\label{sec:Usage}
This section explain how the Qedo code generator can be used.

\section{IFRRepository}
\label{sec:IFRRepository}
The IFR Repository is a hand-written model repository, which is accessible by CORBA interfaces. This IFR is used internally by the code generator and can be started as a stand-alone applications. Is is started by calling: 

\small
\begin{verbatim}

  IFRRepository

\end{verbatim}
\normalsize


If the repository is created successfully it writes the reference to the repository interface into the file \verb repository.ior .

\section{Code Generator}
\label{sec:cidlGen}
This section describes the options of the Qedo code generator.

\subsection{General Options}
\label{sec:GeneralOptions}

There are options which can be used to influence the output of the Qedo compiler tool.
\begin{itemize}
	\item \verb --help : list the short summary of all options

	\item \verb|--out <file_prefix>|: output IDL files will prefixed with \verb <file_prefix> . 

	\item \verb --servant : produces the servant code. This includes a C++ header and implementation file
	
	\item \verb|--business| : This produces Business code skeletons (a header and an implementation file) with user section, which are protected from overwriting if the tool is called multiple times.
	
	\item \verb|-d| : produces XML descriptors

	\item \verb --vc7 : produces Visual Studio project files. Those project files can be included into a solution file zu build the component implementation in a very easy way.

	\item \verb --mkfile : produces a makefiles similar to the Visual Studio project files. By simply calling make the component implementation is built.

	\item \verb --target:  The \verb --target  command line option is used to define the scope of the compilation. Because sometimes there are many components defined in an IDL file but the equivalent and local IDL is only needed for one component. To accomplish this, the target can be a composition name. The target can also be a module name or a component type name. If the target is a module all components within this module are considered for the compilation process.
	
	\item \verb|--feed-ifr <repository.ior>|:  This option starts only the feeder. The Feeder reads the IDL input and stores the model in the IFR Repository. The root reference to the IFR Repository (\verb <repository.ior> ) should be given as a stringified object reference. This option is most likely used without any other output option, since it is only meant to transfers the IDL model into the IFR Model for later use. If other option are used code is generated from the externally used IFR repository.
	
	\item \verb|--read-ifr <repository.ior>|: Using this option omits starting the feeder. It only uses information stored in the IFR. The reference to the IFR shall be provided as stringified object reference (\verb <repository.iro ).
	
\end{itemize}



\subsection{VC7 Project Generator}
\label{sec:VC7ProjectGenerator}

Short instructions to use the code generator

The Qedo code generator is able to generate complete workspace files for the Visual Studio .Net. Since it is not easy to write those files by hand it is strongly recommended to use this feature. After Installing the Qedo runtime, the following steps needs to be done if you start to develop you own CORBA components.

\begin{itemize}
	\item Create a \verb <your_name>.idl  file. 
	In this file you need to define your plain CORBA interfaces as well as the CORBA components. Start the file with an \verb #include \verb  "Components.idl"  statement.

	\item Create a \verb <your_name>.cidl  file.
	In this file you need to specify the compositions that should implement your components. Start the file with a \verb #include  \verb <your_name.idl>  .

	\item Open a command line tool. 
	In this command line tool you need to execute the following command.
For MICO you should use a command similar to this

\small
\begin{verbatim} 

	%QEDO%\bin\cidl_gen -I%QEDO%\idl -I%MICO%\include  
		-I%MICO%\include\mico -DWIN32 -DMICO_ORB -DMICO_CIDL_GEN 
		--vc7 --target CallerImpl hello.cidl
\end{verbatim}
\normalsize

FOR Orbacus you can use a command similar tho this.
\small
\begin{verbatim} 
	%QEDO%\bin\cidl_gen -I%QEDO%\idl -I%ORBACUS%\idl\OB  
		-I%ORBACUS%\idl -DWIN32 -DORBACUS_ORB 
		--vc7 --target CallerImpl hello.cidl
\end{verbatim}
\normalsize

The target could be a composition name or a module name. Depending on the target the code generator will create new directories and within them new project files. For each composition there is a servant and an executor module. The project file for the executor module is depending on the servant module. For this reason it is strongly recommended to keep both project in one solution file.

Furthermore the project file for the executor module contains a post-build step which tries to zip the servant dll and the executor dll into one archive. This post-build step requires a meta-inf directory and two descriptors, namely CORBA Component Descriptor and a Software Package Descriptor. A skeleton version of those descriptors can be created by using the -d option of the generation tool. These files need be copied to the met-inf directory otherwise the post-build step will fail.

	\item Create a solution. Now you can create new (empty) solution with the Visual Studio. Now you can add the needed projects. 

	\item Configuration. Choose the configuration you want to build. This has to correspond to the version of the installed Qedo runtime.

	\item Pre-built.
	To allow the generation of all needed files you will have to build all projects. Depending on your IDL this could cause errors in the executor project. Those errors can be ignored at this stage.

	\item Implement the executors. 
	In the executor project you will now find executor files. The names of these files correspond to the composition description you gave in the \verb <your_name>.cidl  file. Within these files you can see user sections. Do only write your code between an opening and a closing user section statements. All other code is removed at next call of the code generator. If you modify the \verb <your_name>.idl  file you have to re-build the projects. But your code will remain in the implementation files.

	\item Build. Just build all and you will get your components as dlls.

\end{itemize}

\subsection{Makefile Generator}
\label{sec:MakefileGenerator}

Short instructions to use the code generator.

The Qedo code generator is able to generate Makefiles, similar to the project files. These makefile can be used to compile link an package component implementations.
After Installing the Qedo runtime, the following steps needs to be done if you start to develop you own CORBA components on Linux systems.

\begin{itemize}
	\item Create a \verb <your_name>.idl  file. 
	In this file you need to define your plain CORBA interfaces as well as the CORBA components. Start the file with an \verb #include \verb "Components.idl"  statement.

	\item Create a \verb <your_name>.cidl  file.
	In this file you need to specify the compositions that should implement your components. Start the file with a \verb #include  \verb <your_name.idl>  .

	\item Open a shell. 
	In this shell you need to execute the following command.

\small
\begin{verbatim} 

	${QEDO}/bin/cidl_gen -I${QEDO}/idl -I${MICO}/include  
		-I${MICO}/include/mico -DMICO_ORB -DMICO_CIDL_GEN 
		--mkfile --target CallerImpl hello.cidl}
\end{verbatim}
\normalsize

Replace the for example the \verb ${QEDO}  by an appropriate path if needed. Use also the \verb -d  option, if you like to create an initial version of the descriptors.

The target could be a composition name or a module name. It is strongly recommended to use only compositions as target. This will produce separate libraries for the servants and this may cause less trouble at runtime. Depending on the target the code generator will create new directories and within them new Makefiles. For each composition there is a servant and an executor module.

	\item Create a top level Makefile . The top level Makefile is very much dependent on you personal needs. It is recommended to use the \verb MakComponentVars  to make use of a set of pre-defined variables. In the examples module are a lot of top level Makefiles which can be freely (re-)used for that purpose too. The top level Makefile at least needs to call make in the servant and in the executor directory of each relevant component created by the Makefile generator.  

	\item Initial Built. This initial Built will create initial versions of the files. Since the Qedo Code Generator generates skeletons for the business code in some cases just calling make will produce the component libraries. In other case the c++ compile will complain about missing return statements. Now is the time to fill in the business code in the generated skeletons.

	\item Implement the executors. 
	In the executor directory you will now find executor files. The names of these files correspond to the composition description you gave in the \verb <your_name>.cidl  file. Within these files you can see user sections. Do only write your code between an opening and a closing user section statements. All other code is removed at next call of the code generator. If you modify the \verb <your_name>.idl  file you have to re-build the projects. But your code will remain in the implementation files.

	\item Build. Just call make for each directory or preferably by using you own top-level Makefile.

	\item Package. After successfully building the lib they need to get packaged into a component package. For this reason the generated Makefiles of the executor module support the \verb make  \verb package  rule. This will initially create the \verb meta-inf  directory and the CORBA Component Descriptor and the Component Software Descriptor. If these files are already present they are reserved. This files need some modification. Not all information needed for these files can be generated out of the IDL/CIDL description.
	\item Build Assembly. Consult the examples module to see how an assembly is built.
\end{itemize}

\end{document}

